\documentclass[runningheads,a4paper]{llncs}

\usepackage{amssymb}
\setcounter{tocdepth}{3}
\usepackage{graphicx}
\usepackage{amsmath}
\usepackage{verbatim}
\usepackage[margin=0.9in]{geometry}
%\usepackage{amsfonts}
%\usepackage{amsthm}
\usepackage{subfigure}
\usepackage{mathtools}
%\usepackage{caption}
%\usepackage{subcaption}
%\usepackage{cite}
\usepackage{hyperref}
\usepackage{url}
\urlstyle{same}
\newcommand{\keywords}[1]{\par\addvspace\baselineskip
\noindent\keywordname\enspace\ignorespaces#1}

\makeatletter
\let\c@lemma=\c@theorem
\let\c@corollary=\c@theorem
\let\c@fact=\c@theorem
\makeatother

\let\realendproof=\endproof
\def\endproof{\hspace*{\fill}$\Box$\realendproof}
\title{Self Referencing Sequences}
\author{Perry Kleinhenz, Fermi Ma and Erik Waingarten}
\date{}							% Activate to display a given date or no date

\begin{document}

\title{Self Referencing Sequences}
\titlerunning{Self Referencing Sequences}

\author{Perry Kleinhenz \and Fermi Ma \and Erik Waingarten}
%
\authorrunning{Perry Kleinhenz \and Fermi Ma \and Erik Waingarten}
% (feature abused for this document to repeat the title also on left hand pages)

% the affiliations are given next; don't give your e-mail address
% unless you accept that it will be published
\institute{
\protect\url{{pkleinhe, fermima,eaw}@mit.edu}}

\maketitle

\section{Introduction}

\emph{Written by Fermi Ma, edited by Perry Kleinhenz and Erik Waingarten}

\subsection{Problem Setup}
Consider the sequence
\begin{align*}
1,2,2,1,1,2,1,2,2,1,2,2,1,1,2,1,1,2,2,\dots
\end{align*}
We can break up the sequence into contiguous blocks, where each block is a stretch of repeated numbers:
\begin{align*}
[1],[2,2],[1,1],[2],[1],[2,2],[1],[2,2],[1,1],[2],[1,1],[2,2],\dots
\end{align*}
The block lengths, read from left to right, reproduce the original sequence. Thus, this sequence is \emph{self-referential}, and for the remainder of this paper we study properties of such sequences. 

We say that this sequence is \emph{generated} by the set of numbers used in the sequence, so the above sequence is generated by $\{1,2\}$. We will refer to this set as the \emph{generating set}. Observe that it is unambiguous which number to use next. After a block of 1's, the next block must be a block of 2's. However, in a sequence with generating set $\{1,2,3\}$, for example, a block of 1's could be followed by either a block of 2's or a block of 3's. Thus, we will assume that the next number to use is the following element (in cyclic order) in the generating set. (Note that because order matters, the \emph{generating set} is technically not a set, but we do not call it a ``sequence" to avoid possible confusion with its corresponding self-referencing sequence.)

With this rule, a self-referencing sequence generated by $\{1,2,3\}$ has blocks of 1's followed by blocks of 2's followed by blocks of 3's, followed by blocks of 1's, and so on. Thus, $\{1,2,3\}$ can generate
\begin{align*}
1,2,2,3,3,1,1,1,2,2,2,3,2,1,2,2,3,3,1,1,\dots
\end{align*}
but not
\begin{align*}
1,3,3,3,2,2,2,1,1,1,3,3,2,2,1,1,3,2,1,3,\dots
\end{align*}
as the second sequence has the numbers in the wrong order.

\subsection{Overview}

In this paper, we address a variety of questions regarding these sequences. In Section~\ref{determinism} we show that if we are given a generating set and a starting number for the sequence, the sequence is uniquely determined. After that, we consider how the starting number for the sequence affects what the sequence can be. In particular, in Section~\ref{prefixes}, we show that there exist limits on how much two sequences with the same generating set can resemble each other if they start with different numbers. In the following Section~\ref{equivalence}, we consider the limiting behavior of such sequences, and we conjecture that the limiting behavior of a sequence is determined only by the generating set, and not the starting number. 

In Section~\ref{iterativeprocess}, we take a different approach to analyzing these sequences by showing that there is an alternate way of generating them with an iterative process. In Section~\ref{substitutionrules}, we show how a large number of these sequences can actually be generated with a simple set of substitution rules. We then introduce the density problem in Section~\ref{density}, and use ideas developed in Section~\ref{iterativeprocess} and Section~\ref{substitutionrules} to answer the question for certain types of generating sets. Unfortunately, we find that the open problem of determining the density for the sequence of 1's and 2's is hard to solve using our methods.

\section{Determinism}
\label{determinism}

\emph{Written by Fermi Ma, edited by Perry Kleinhenz and Erik Waingarten}

We claim that a self-referencing sequence is uniquely determined by the generating set and the starting number. To see why this is true, we can simply give an algorithm to build the sequence. Assume the generating set is $\{a_1,a_2,\dots,a_m\}$ and that we start with some $a_i$.

\textbf{Algorithm:}
Assume that the sequence $\{s\}$ is initially just $a_i$. At the next step, add an additional $a_i -1$ repetitions of $a_i$ to the end of the sequence. Now if there is a second term in the sequence, read that term and write down that many copies of $a_{i+1}$. If there is no term, write down $a_{i+1}$.

NEED TO FINISH THIS. 

Since we know that a self-referencing sequence is uniquely determined by the generating set and the starting number, we introduce the notation $s(A,a)$ to denote the unique sequence generated by set $A$ with starting number $a \in A$. 

\section{Prefixes}
\label{prefixes}
\emph{Written by Perry Kleinhenz, edited by Fermi Ma and Erik Waingarten}
\begin{definition} We say that two sequences $\{a_i\}$ and $\{b_i\}$ differ by a prefix of length $n$ if 
\begin{equation*}
a_{n+i} = b_{i},
\end{equation*}
for $i$ any positive integer. if no such $n$ exists then we say that the two sequences are independent.
\end{definition}
Note that a pair of sequences can have prefixes of different length. For example if we have 
\begin{align*}
&\{a_i\} = \{1,2,3,4,5,6,1,2,3,4,5,6, \ldots \} \\  
&\{b_i\}= \{7,8,9,1,2,3,4,5,6,1,2,3,4,5,6, \ldots\},
\end{align*}
 then $a_i$ and $b_i$ differ by prefixes of length $3+6k$ for $k$ a nonnegative integer. Because of this we make the following definition.
 \begin{definition} 
 We say that two sequences $\{a_i\}$ and $\{b_i\}$ differ by a minimal prefix of length $n$ if they differ by a prefix of length $n$, but do not differ by a prefix of length $m$ for all $0<m<n$.
 \end{definition}
We note that if a pair of sequences differs by a prefix they differ by a minimal prefix. In our above example $a_i$ and $b_i$ differ by a minimal prefix of length $3$. 
 
\begin{theorem} The self-referencing sequence beginning with a $1$ generated over $\{1, c_1, c_2, \ldots, c_n\}$ differs from the self-referencing sequence beginning with $c_1$ generated over $\{1, c_1, c_2, \ldots, c_n\}$ by a minimal prefix of length 1.
\end{theorem}
\begin{proof}
Let $a_i$ refer to the $i$th term of the sequence beginning with a $1$ and $b_i$ refer to the $i$th term of the sequence beginning with $c_1$. We know that $a_1=1$, this means that the first block in $a_i$ must be of length 1. In other words we must have $a_2=c_1$, but since $a_2$ is now part of the second block of $a_i$ the value of $a_3$ is not specified by $a_1$. Thus we could specify the rest of $a_i$ only knowing the value of $a_2$, but $b_1=c_1$, so this is exactly how the values of $b_i$ is specified. Therefore
\begin{equation*}
a_{1+i} = b_{i},
\end{equation*}
so $a_i$ and $b_i$ differ by a prefix of length 1, but $a_{1} \neq b_{1}$ so it is a minimal prefix. 
\end{proof}

\begin{theorem}
\label{independentsequences}
If $1<a<b$ then the self-referencing sequence beginning with $a$ generated over $\{1, a, b\}$ and the self-referencing sequence beginning with $b$ generated over $\{1, a, b\}$ are independent.
\end{theorem}
\begin{proof}
We proceed by contradiction. If the two sequences are not independent then they must differ by some minimal prefix of length $n$. If we let $a_i$ be the sequence beginning with $a$ and $b_i$ be the sequence beginning with $b$ the
\begin{equation*}
a_{n+k} = b_{k}.
\end{equation*}
We know that the first $b$ terms of the sequence which begins with a $b$ are $b$'s. Because $b>a>1$ no block can be longer then $b$. Thus the terms immediately preceding and following this block in the sequence $a_i$ must not be $b$'s. Because of this the first block of $b_i$ is the $m+1$th block of $a_i$, where $m$ is some positive integer. Since the value of the sequence at position $k$ is also the length of the $k$th block this means that $a_{m+1}=b_1$. In fact the $k$th block of $b_i$ is the $(m+k)$th block of $a_i$ and so 
\begin{equation*}
a_{m+k} = b_k
\end{equation*}

We now claim that the prefix has length strictly larger than $m$. Assume otherwise so $n \leq m$. 
This means the entry in the $a_i$ which gives the length of the $m+1$th block occurs at or after the beginning of the $m+1$th block. By our proof that self-referential sequences are well defined we know that the second case cannot occur. Therefore the only other option would be for the $m+1$th block to start at position $m+1$, but this would require every block to have length exactly 1, but this is a contradiction because $a_1=a>1$, that is the first block has length greater than one. Therefore the prefix must have length strictly larger than $m$. 

We note that based on a previous step the two sequences differ by a prefix starting at $m$. Since $m<n$ we have produced a prefix which starts before $n$ which is a contradiction. Therefore the two sequences are independent. 
\end{proof}

\section{Equivalence}
\label{equivalence}
\emph{Written by Fermi Ma, edited by Perry Kleinhenz and Erik Waingarten}

In this section, we consider a slightly weaker definition of equivalence. Consider, for example, sequences generated by $\{1,2,3\}$. Theorem~\ref{independentsequences} shows that the sequence starting with 2
\begin{align*}
2,2,3,3,1,1,1,2,2,2,3,1,2,3,3,1,1,2,2,3,3,3,\dots
\end{align*}
and the sequence starting with 3
\begin{align*}
3,3,3,1,1,1,2,2,2,3,1,2,3,3,1,1,2,2,3,3,3,\dots
\end{align*}
are independent in the sense that neither sequence is exactly contained in the other. However, if we delete the first two numbers of the sequence starting with 2, and the first number of the sequence starting with 3, we get 
\begin{align*}
3,3,1,1,1,2,2,2,3,1,2,3,3,1,1,2,2,3,3,3,\dots
\end{align*}
in both cases. (Need to further justify why these two are the same, probably with a reading distance argument)

Thus, since both sequences are essentially the same if certain prefixes are removed, we call these sequences \emph{equivalent}. To formalize, this, we say that two sequences $\{a_i\}$ and $\{b_i\}$ are equivalent if there exists some integers $n,k \geq 0$ such that
\begin{align*}
a_{n+i} = b_{k+i}
\end{align*}

We now give the following conjecture

\begin{conjecture} Any sequence generated by a set $S$ is equivalent to any other sequence generated by the same set $S$.
\end{conjecture}

This conjecture is partly motivated by the fact that the property holds for $\{1,2,3\}$. For generating sets with larger numbers, we have been unable to verify or refute the conjecture. However, in Section~\ref{Density}, we show that the starting number of a sequence does not affect the densities of each number, which provides some evidence for this conjecture.

\section{An Iterative Process}
\label{iterativeprocess}

\emph{Written by Fermi Ma, edited by Perry Kleinhenz and Erik Waingarten}

We change focus in this section and consider how to generate self-referential sequences with an iterative process. We first look at the sequences generated by $\{1,2\}$.

One possible iterative process is as follows. Let the sequence $\{s^0\}$ be 2, and let the sequence $\{s^{(i)}\}$ for $i \geq 1$ be the unique sequence that starts with 2 and is such that its block lengths, read from left to right, reproduce the sequence $\{s^{(i-1)}\}$. This gives:
\begin{align*}
\{s^{(0)}\} &= 2\\
\{s^{(1)}\} &= 2,2\\
\{s^{(2)}\} &= 2,2,1,1\\
\{s^{(3)}\} &= 2,2,1,1,2,1\\
\{s^{(4)}\} &= 2,2,1,1,2,1,2,2,1\\
&\cdots
\end{align*}
Let $s^\infty$ be the limit of this process. We claim that $s^\infty$ is the unique self-referential sequence $s(\{1,2\},2)$ generated by $\{1,2\}$ with starting number 2.
\begin{proof}
We show that this is true by simply showing that if $s^{(i)}$ is a prefix of
\end{proof}

\section{Substitution Rules}
\label{substitutionrules}

\section{Density}
\label{density}

\emph{Written by Fermi Ma, edited by Perry Kleinhenz and Erik Waingarten}

The above analysis shows that, for example, the s

\end{document}  
